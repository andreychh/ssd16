% (The MIT License)
%
% Copyright (c) 2021 Yegor Bugayenko
%
% Permission is hereby granted, free of charge, to any person obtaining a copy
% of this software and associated documentation files (the 'Software'), to deal
% in the Software without restriction, including without limitation the rights
% to use, copy, modify, merge, publish, distribute, sublicense, and/or sell
% copies of the Software, and to permit persons to whom the Software is
% furnished to do so, subject to the following conditions:
%
% The above copyright notice and this permission notice shall be included in all
% copies or substantial portions of the Software.
%
% THE SOFTWARE IS PROVIDED 'AS IS', WITHOUT WARRANTY OF ANY KIND, EXPRESS OR
% IMPLIED, INCLUDING BUT NOT LIMITED TO THE WARRANTIES OF MERCHANTABILITY,
% FITNESS FOR A PARTICULAR PURPOSE AND NONINFRINGEMENT. IN NO EVENT SHALL THE
% AUTHORS OR COPYRIGHT HOLDERS BE LIABLE FOR ANY CLAIM, DAMAGES OR OTHER
% LIABILITY, WHETHER IN AN ACTION OF CONTRACT, TORT OR OTHERWISE, ARISING FROM,
% OUT OF OR IN CONNECTION WITH THE SOFTWARE OR THE USE OR OTHER DEALINGS IN THE
% SOFTWARE.

\documentclass{article}
\usepackage{../ssd}
\usepackage{../inno}
\usepackage{../slides}
\newcommand*\thetitle{XML}
\newcommand*\thesubtitle{vs JSON, YAML, TOML, etc.}
\begin{document}

\plush{\innoTitlePage{7}}

\innoToc

\plush{\innoChapter[XML]{Extensible Markup Language (XML)}}

\subcrumbection{XML}
\plush[4]{
  \innoSection{Library in XML}
  \innoSnippet{library.xml}
}

\subcrumbection{Namespaces}
\plush[3]{
  \innoSection{Namespaces}
  \innoSnippet{namespaces.xml}
}

\subcrumbection{Escaping}
\plush[2]{
  \innoSection{Escaping}
  \innoSnippet{escaping.xml}
}

\subcrumbection{Formats}
\plush[4]{
  \innoPinQR{https://en.wikipedia.org/wiki/Category:XML-based\_standards}
  \innoSection{XML Based Formats/Protocols}
  SOAP, RSS, Atom, SVG, XHTML, HTML5, \\
  Open Office XML, XMPP, \\
  SyncML, RDF, XMI, XMIR :)
}

\plush{\innoChapter[XSD]{XSD, XPath, XSLT, XQuery, etc.}}

\subcrumbection{XSD}
\plush[4]{
  \innoSection{XML Schema Definition (XSD)}
  \innoSnippet[\scriptsize]{library.xsd}
}

\subcrumbection{XPath}
\plush[4]{
  \innoSection{XML Path Language (XPath)}
  \ff{<library><book id=42><author>David West</..></..></..>}\par
  \vspace*{12pt}
  \ff{/library/book[@id='42']}\par
  \ff{//book[@id='42']}\par
  \ff{//book[first()]}\par
  \ff{//book[author='David West']}\par
  \ff{//book[author[text()='David West']]}
}

\subcrumbection{XSL}
\plush[4]{
  \innoSection{XSL Transformations (XSLT)}
  \innoSnippet[\scriptsize]{library.xsl}
}

\plush{\innoChapter[JSON]{JavaScript Object Notation (JSON)}}

\subcrumbection{JSON}
\plush[3]{
  \innoPinQR{https://www.yegor256.com/2015/11/16/json-vs-xml.html}
  \innoSection{JSON for the Library}
  \innoSnippet{library.json}
}

\plush[2]{
  \innoBanner{JSON to JavaScript Object and Backwards}
  \ff{var a = JSON.parse('\{"age": 25\}').age;}\par
  \ff{JSON.stringify(\{age: 25\});}
}

\plush{\innoChapter[Others]{YAML, TOML, CSV}}

\subcrumbection{YAML}
\plush[2]{
  \innoSection{Yet Another Markup Language (YAML)}
  \innoSnippet{library.yml}
}

\subcrumbection{TOML}
\plush[2]{
  \innoSection{TOML}
  \innoSnippet{library.toml}
}

\subcrumbection{CSV}
\plush[2]{
  \innoPinQR{https://en.wikipedia.org/wiki/List\_of\_file\_formats}
  \innoSection{Comma-Separated Values (CSV)}
  \innoSnippet{library.csv}
}

\innoBVC

\plush{
  \begin{multicols}{2}
    \innoBook{xml-nutshell}
      {\nospell{Elliotte Rusty Harold} et al.}
      {XML in a Nutshell, Third Edition}
    \par\columnbreak
    \innoBook{xslt}
      {\nospell{Michael James Fitzgerald}}
      {Learning XSLT: A Hands-On Introduction to XSLT and XPath}
  \end{multicols}
}

\plush[2]{
  \innoBanner{Call to Action:}
  In your application, make sure your data is represented in
  XML, at least in one place, and being transformed by XSLT.\par
  Design your own data format.
}

\plush[3]{
  \innoBanner[orange]{Still unresolved issues:}
  \begin{itemize}
    \item How to \ul{map} XML/JSON to objects?
    \item How to \ul{print} object to XML/JSON?
    \item How to \ul{create} a common binary format?
    \item How to \ul{restore} the popularity of XSLT?
  \end{itemize}
}

\end{document}
