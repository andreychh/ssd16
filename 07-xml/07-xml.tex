% SPDX-FileCopyrightText: Copyright (c) 2021 Yegor Bugayenko
% SPDX-License-Identifier: MIT

\documentclass{article}
\usepackage{../ssd}
\newcommand*\thetitle{XML}
\newcommand*\thesubtitle{vs.\ JSON, YAML, TOML, etc.}
\begin{document}

\lnTitlePage{7}{16}{yz0nItCU-xQ}

\pptToc

\plush{\pptChapter[XML]{Extensible Markup Language (XML)}}

\begin{lnSnippet}[library.xml]
<?xml version="1.0" encoding="UTF-8"?>
<library>
  <book id="42">
    <author>David West</author>
    <title>Object Thinking</title>
  </book>
  <book id='43'>
    <author>Martin Fowler</author>
    <title>Refactoring</title>
  </book>
</library>
\end{lnSnippet}
\plush[4]{%
  \pptSection[XML]{Library in XML}
  \small\ffinput{library.xml}
}

\begin{lnSnippet}[namespaces.xml]
<?xml version="1.0" encoding="UTF-8"?>
<library xmlns="https://innopolis.university/ssd16"
  xmlns:a="https://www.amazon.com"
  xmlns:t="https://www.twitter.com">
  <book id="42">
    <a:dp>0134757599</a:dp>
    <t:author>@martinfowler</t:author>
    <author>Martin Fowler</author>
    <title>Refactoring</title>
  </book>
</library>
\end{lnSnippet}
\plush[3]{%
  \pptSection{Namespaces}
  \small\ffinput{namespaces.xml}
}

\begin{lnSnippet}[escaping.xml]
<?xml version="1.0" encoding="UTF-8"?>
<formulas>
  <f title='Fibonacci&apos;s'> <!-- Fibonacci's -->
    <e>if x &lt; 2 return x</e> <!-- if x < 2 return x -->
    <e>else return f(x-1) + f(x-2)</e>
  </f>
</formulas>
\end{lnSnippet}
\plush[2]{%
  \pptSection{Escaping}
  \small\ffinput{escaping.xml}
}

\plush[4]{%
  \pptPinQR{https://en.wikipedia.org/wiki/Category:XML-based\_standards}
  \pptSection[Formats]{XML Based Formats/Protocols}
  SOAP, RSS, Atom, SVG, XHTML, HTML5, \\
  Open Office XML, XMPP, \\
  SyncML, RDF, XMI, XMIR :)
}

\plush{\pptChapter[XSD]{XSD, XPath, XSLT, XQuery, etc.}}

\begin{lnSnippet}[library.xsd]
<?xml version="1.0" encoding="UTF-8"?>
<xs:schema xmlns:xs="http://www.w3.org/2001/XMLSchema">
  <xs:complexType name="book">
    <xs:sequence>
      <xs:element name="author" minOccurs="1" maxOccurs="1"/>
      <xs:element name="title" minOccurs="1" maxOccurs="1"/>
    </xs:sequence>
    <xs:attribute name="id" type="xs:decimal"/>
  </xs:complexType>
  <xs:element name="library">
    <xs:complexType>
      <xs:sequence>
        <xs:element name="book" type="book" minOccurs="0"/>
      </xs:sequence>
    </xs:complexType>
  </xs:element>
</xs:schema>
\end{lnSnippet}
\plush[4]{%
  \pptSection[XSD]{XML Schema Definition (XSD)}
  \scriptsize\ffinput{library.xsd}
}

\plush[4]{%
  \pptSection[XPath]{XML Path Language (XPath)}
  \ff{<library><book id=42><author>David West</..></..></..>}\par
  \vspace*{12pt}
  \ff{/library/book[@id='42']}\par
  \ff{//book[@id='42']}\par
  \ff{//book[first()]}\par
  \ff{//book[author='David West']}\par
  \ff{//book[author[text()='David West']]}
}

\begin{lnSnippet}[library.xsl]
<?xml version="1.0" encoding="UTF-8"?>
<xsl:stylesheet version="2.0"
  xmlns:xsl="http://www.w3.org/1999/XSL/Transform">
  <xsl:template match="book">
    <item>
      <xsl:value-of select="title"/>
      <xsl:text> by </xsl:text>
      <xsl:value-of select="title"/>
    </item>
  </xsl:template>
  <xsl:template match="node()|@*">
    <xsl:copy>
      <xsl:apply-templates select="node()|@*"/>
    </xsl:copy>
  </xsl:template>
</xsl:stylesheet>
\end{lnSnippet}
\plush[4]{%
  \pptSection[XSL]{XSL Transformations (XSLT)}
  \scriptsize\ffinput{library.xsl}
}

\plush{\pptChapter[JSON]{JavaScript Object Notation (JSON)}}

\begin{lnSnippet}[library.json]
[
  {
    "author": "David West",
    "id": 42,
    "title": "Object Thinking"
  },
  {
    "author": "Martin Fowler",
    "id": 43,
    "title": "Refactoring"
  }
]
\end{lnSnippet}
\plush[3]{%
  \pptPinQR{https://www.yegor256.com/2015/11/16/json-vs-xml.html}
  \pptSection[JSON]{JSON for the Library}
  \small\ffinput{library.json}
}

\plush[2]{%
  \pptBanner{JSON to JavaScript Object and Backwards}
  \ff{var a = JSON.parse('\{"age": 25\}').age;}\par
  \ff{JSON.stringify(\{age: 25\});}
}

\plush{\pptChapter[Others]{YAML, TOML, CSV}}

\begin{lnSnippet}[library.yml]
library:
- id: 42
  author: David West
  title: Object Thinking
- id: 43
  author: Martin Fowler
  title: Refactoring
\end{lnSnippet}
\plush[2]{%
  \pptSection[YAML]{Yet Another Markup Language (YAML)}
  \small\ffinput{library.yml}
}

\begin{lnSnippet}[library.toml]
[library.a]
  id = 42
  author = "David West"
  title = "Object Thinking"
[library.b]
  id = 43
  author = "Martin Fowler"
  title = "Refactoring"
\end{lnSnippet}
\plush[2]{%
  \pptSection{TOML}
  \small\ffinput{library.toml}
}

\begin{lnSnippet}[library.csv]
Id,Author,Title
42,David West,Object Thinking
43,"Martin Fowler","Refactoring"
\end{lnSnippet}
\plush[2]{%
  \pptPinQR{https://en.wikipedia.org/wiki/List\_of\_file\_formats}
  \pptSection[CSV]{Comma-Separated Values (CSV)}
  \small\ffinput{library.csv}
}

\plush{\innoBVC}

\plush{%
  \begin{multicols}{2}
    \innoBook{harold2000xml}
      {\nospell{Elliotte Rusty Harold} et al.}
      {XML in a Nutshell, Third Edition}
    \par\columnbreak
    \innoBook{xslt}
      {\nospell{Michael James Fitzgerald}}
      {Learning XSLT: A Hands-On Introduction to XSLT and XPath}
  \end{multicols}
}

\plush[2]{%
  \pptBanner{Call to Action:}
  In your application, make sure your data is represented in
  XML, at least in one place, and being transformed by XSLT.\par
  Design your own data format.
}

\plush[3]{%
  \pptBanner[orange]{Still unresolved issues:}
  \begin{itemize}
    \item How to \ul{map} XML/JSON to objects?
    \item How to \ul{print} object to XML/JSON?
    \item How to \ul{create} a common binary format?
    \item How to \ul{restore} the popularity of XSLT?
  \end{itemize}
}

\end{document}
