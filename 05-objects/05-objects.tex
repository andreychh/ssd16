% (The MIT License)
%
% Copyright (c) 2021 Yegor Bugayenko
%
% Permission is hereby granted, free of charge, to any person obtaining a copy
% of this software and associated documentation files (the 'Software'), to deal
% in the Software without restriction, including without limitation the rights
% to use, copy, modify, merge, publish, distribute, sublicense, and/or sell
% copies of the Software, and to permit persons to whom the Software is
% furnished to do so, subject to the following conditions:
%
% The above copyright notice and this permission notice shall be included in all
% copies or substantial portions of the Software.
%
% THE SOFTWARE IS PROVIDED 'AS IS', WITHOUT WARRANTY OF ANY KIND, EXPRESS OR
% IMPLIED, INCLUDING BUT NOT LIMITED TO THE WARRANTIES OF MERCHANTABILITY,
% FITNESS FOR A PARTICULAR PURPOSE AND NONINFRINGEMENT. IN NO EVENT SHALL THE
% AUTHORS OR COPYRIGHT HOLDERS BE LIABLE FOR ANY CLAIM, DAMAGES OR OTHER
% LIABILITY, WHETHER IN AN ACTION OF CONTRACT, TORT OR OTHERWISE, ARISING FROM,
% OUT OF OR IN CONNECTION WITH THE SOFTWARE OR THE USE OR OTHER DEALINGS IN THE
% SOFTWARE.

\documentclass[white]{../slidedeck}
\newcommand*\thetitle{Object Thinking}
\newcommand*\thesubtitle{and Domain Driven Design}
\begin{document}

\innoTitlePage{5}

\sdPrintAndFlush{\sdQuote{edsger-dijkstra}{Object-oriented programs are offered as alternatives to correct ones}{\nospell{Edsger W. Dijkstra} (1989)}}

\sdPrintAndFlush{\sdQuote{alan-kay}{I invented the term \emph{object-oriented}, and I can tell you I did not have C++ in mind}{\nospell{Alan Kay} (1997)}}

\sdPrintAndFlush{\sdQuote{paul-graham}{Object-oriented programming offers a sustainable way to write spaghetti code}{\nospell{Paul Graham} (2003)}}

\sdPrintAndFlush{\sdQuote{linus-torvalds}{C++ is a horrible language. C++ leads to really, really bad design choices. ... idiotic object model crap.}{\nospell{Linus Torvalds} (2007)}}

\sdTocPrint

\sdToc{Theory}{The Philosophy of OOP}

\sdStoc{GOTO}
\sdPrintAndFlush[3]{
  \sdBanner{The Era of GOTO}
  \sdPic{0.7}{basic}
}

\sdStoc{IF/THEN}
\sdPrintAndFlush[3]{
  \sdBanner{Structured Programming}
  \begin{multicols}{2}
    \sdPic{0.9}{pascal}
    \par\columnbreak
    \sdPic{0.7}{flowchart}
  \end{multicols}
}

\sdStoc{CALL}
\sdPrintAndFlush[3]{
  \sdBanner{Procedural Programming}
  \sdPic{0.5}{calls}
}

\sdStoc{OOP$_1$}
\sdPrintAndFlush[3]{
  \sdBanner{Object-Oriented Programming ... Not!}
  \sdPic{0.5}{oop-bad}
}

\sdPrintAndFlush[3]{
  \sdBanner{Spaghetti OOP Code}
  \sdPic{0.8}{rectangles1}
}

\sdStoc{OOP$_2$}
\sdPrintAndFlush[3]{
  \sdBanner{OOP Done Right}
  \sdPic{0.6}{oop-good}
}

\sdPrintAndFlush[3]{
  \sdBanner{Elegant OOP Code}
  \sdPic{0.8}{rectangles2}
}

\sdToc{Object}{What is an Object?}

\sdStoc{C++}
\sdPrintAndFlush{\sdCite{stroustrup}{An object is some memory that holds a value of some type}{\emph{Programming Principles and Practice Using C++} by \nospell{Bjarne Stroustrup}}}

\sdStoc{Wiki}
\sdPrintAndFlush{\sdCite{wikipedia}{Objects may contain data, in the form of fields, often known as attributes; and code, in the form of procedures, often known as methods}{Wikipedia}}

\sdStoc{Smalltalk}
\sdPrintAndFlush{\sdCite{smalltalk}{An object consists of some private memory and a set of operations}{\emph{Smalltalk-80: The Language and Its Implementation} by \nospell{Adele Goldberg} et al., p.~6}}

\sdStoc{Java}
\sdPrintAndFlush{\sdCite{java-nutshell}{A class is a collection of data fields that hold values and methods that operate on those values}{\emph{Java in a Nutshell} by \nospell{Ben Evans}}}

\sdStoc{\nospell{Eckel}}
\sdPrintAndFlush{\sdCite{eckel}{Each object looks quite a bit like a little computer --- it has a state, and it has operations that you can ask it to perform}{\emph{Thinking in Java} by \nospell{Bruce Eckel}, p.~16}}

\sdStoc{West}
\sdPrintAndFlush{\sdCite{object-thinking}{An object is the equivalent of the quanta from which the universe
is constructed}{\emph{Object Thinking} by \nospell{David West}, p.~66}}

\sdToc{Evil}{Three Most Evil Parts of OOP}

\sdStoc{Static}
\sdPrintAndFlush[5]{
  \innoPinQR{https://www.yegor256.com/2014/05/05/oop-alternative-to-utility-classes.html}
  \sdBanner{1. Static Methods}
  \begin{multicols}{2}
    \sdPic{0.9}{static1}
    \par\columnbreak
    \sdPic{0.9}{static2}
  \end{multicols}
}

\sdStoc{Mutability}
\sdPrintAndFlush[5]{
  \innoPinQR{https://www.yegor256.com/2014/11/07/how-immutability-helps.html}
  \sdBanner{2. Mutability vs. Immutability}
  \begin{multicols}{2}
    \sdPic{0.9}{mutable}
    \par\columnbreak
    \sdPic{0.9}{immutable}
  \end{multicols}
}

\sdPrintAndFlush[3]{
  \innoPinQR{https://www.yegor256.com/2014/06/09/objects-should-be-immutable.html}
  \sdBanner{Benefits of Immutability}
  \sdPic{0.8}{immutability-benefits}
}

\sdStoc{NULL}
\sdPrintAndFlush[4]{
  \innoPinQR{https://www.yegor256.com/2014/05/13/why-null-is-bad.html}
  \sdBanner{3. NULL References}
  \sdPic{0.8}{null}\par
  \emph{Null References, The Billion Dollar Mistake}\\
  presentation by \nospell{Tony Hoare},
  \href{http://www.infoq.com/presentations/Null-References-The-Billion-Dollar-Mistake-Tony-Hoare}{watch here}.
}

\sdPrintAndFlush[3]{
  \sdBanner{NULL Object}
  \sdPic{0.8}{null-object}
}

\sdPrintAndFlush[7]{
  \innoPinQR{https://www.yegor256.com/2015/08/25/fail-fast.html}
  \sdBanner{Fail Fast vs. Fail Safe}
  \sdPic{0.7}{fail-fast}
}

\sdToc{DDD}{Domain Driven Design}

\sdPrintAndFlush[4]{
  \sdBanner{Names of Objects Done Right}
  \sdPic{0.6}{ddd}
}

\sdToc{EO}{Elegant Objects}

\sdPrint{\innoPinQR{https://www.elegantobjects.org}}
\sdPrint{\sdBanner{Elegant Objects (EO)}}\sdClick
\sdPrint{Started in 2014}\sdClick
\sdPrint{Two books, 40+ speeches, 80+ blog posts}\sdClick
\sdPrint{30+ frameworks and libraries}\sdClick
\sdPrint{50+ fans registered}\sdClick
\sdPrint{Six bloggers\\\small e.g. \nospell{pragmaticobjects.com}, \nospell{g4s8.wtf}, \nospell{amihaiemil.com}}\sdClick
\sdPrint{Five ``Object Thinking'' Meetups}
\sdFlush

\sdPrintAndFlush[3]{
  \small
  \begin{multicols}{2}
  \emph{Object-Oriented Lies}\\
  JPoint Student Day\\
  Moscow, Russia, 22-24 April 2016\\
  \sdQR[1in]{https://www.youtube.com/watch?v=F4N25kZ2zQU}

  \emph{Java vs. OOP}\\
  JavaDay 2016 \\
  Minsk, Belarus, 11~June~2016\\
  \sdQR[1in]{https://www.youtube.com/watch?v=6hOBfjJ2bpw}

  \par\columnbreak

  \emph{Java vs. OOP}\\
  JavaDay Kyiv 2016 \\
  Kyiv, Ukraine, 15 October 2016\\
  \sdQR[1in]{https://www.youtube.com/watch?v=cGcCcxx4xrg}

  \emph{What's Wrong With OOP?}\\
  RigaDevDays 2017\\
  Riga, Latvia, 15 May 2017\\
  \sdQR[1in]{https://www.youtube.com/watch?v=K_QEOtYVQ7A}
  \end{multicols}
}

\sdPrintAndFlush[5]{
  \innoPinQR{https://www.eolang.org}
  \sdBanner{EOLANG: Our New Programming Language}
  \sdPic{0.5}{eo-sample}
}

\sdPrint{\sdBanner{If you want to help:}}
\sdPrint{EO to JavaScript/Go/Rust/Ruby compilers}\sdClick
\sdPrint{REPL for EO}\sdClick
\sdPrint{Static analysis of EO code}\sdClick
\sdPrint{EO integration with Java/C++}\sdClick
\sdPrint{Automated refactoring of EO code}\sdClick
\sdPrint{JetBrains plugin for EO}
\sdFlush

\innoBVC

\sdPrintAndFlush[3]{
  \begin{multicols}{2}
    \innoBook{object-thinking}
      {David West}
      {Object Thinking}
    \par\columnbreak
    \innoBook{ddd}
      {Eric Evans}
      {Domain-Driven Design: Tackling Complexity in the Heart of Software}
  \end{multicols}
}

\sdPrintAndFlush[3]{
  \begin{multicols}{2}
    \innoBook{elegant-objects-1}
      {Yegor Bugayenko}
      {Elegant Objects, vol. 1}
    \par\columnbreak
    \innoBook{elegant-objects-2}
      {Yegor Bugayenko}
      {Elegant Objects, vol. 2}
  \end{multicols}
}

\sdPrintAndFlush{
  \sdBanner{Where to publish:}\par
  SPLASH: ACM SIGPLAN conference on Systems, Programming, Languages, and Applications\par
  PLDI: ACM SIGPLAN Conference on Programming Language Design and Implementation\par
  POPL: The Annual Symposium on Principles of Programming Languages
}

\sdPrintAndFlush[3]{
  \sdBanner{Call to Action:}
  Take \ff{yegor256/hangman} repository and rewrite it in true
  object-oriented manner, submit a pull request with your changes.
}

\sdPrintAndFlush[3]{
  \sdBanner[orange]{Still unresolved issues:}\par
  \begin{itemize}
    \item How to \ul{motivate} coders for better OO practices?
    \item How to \ul{create} better OO programming languages?
    \item How to \ul{catch} bad OO practices automatically?
    \item How to \ul{prove} some OO practices are bad?
  \end{itemize}
}

\end{document}
