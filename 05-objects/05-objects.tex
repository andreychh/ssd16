% (The MIT License)
%
% Copyright (c) 2021 Yegor Bugayenko
%
% Permission is hereby granted, free of charge, to any person obtaining a copy
% of this software and associated documentation files (the 'Software'), to deal
% in the Software without restriction, including without limitation the rights
% to use, copy, modify, merge, publish, distribute, sublicense, and/or sell
% copies of the Software, and to permit persons to whom the Software is
% furnished to do so, subject to the following conditions:
%
% The above copyright notice and this permission notice shall be included in all
% copies or substantial portions of the Software.
%
% THE SOFTWARE IS PROVIDED 'AS IS', WITHOUT WARRANTY OF ANY KIND, EXPRESS OR
% IMPLIED, INCLUDING BUT NOT LIMITED TO THE WARRANTIES OF MERCHANTABILITY,
% FITNESS FOR A PARTICULAR PURPOSE AND NONINFRINGEMENT. IN NO EVENT SHALL THE
% AUTHORS OR COPYRIGHT HOLDERS BE LIABLE FOR ANY CLAIM, DAMAGES OR OTHER
% LIABILITY, WHETHER IN AN ACTION OF CONTRACT, TORT OR OTHERWISE, ARISING FROM,
% OUT OF OR IN CONNECTION WITH THE SOFTWARE OR THE USE OR OTHER DEALINGS IN THE
% SOFTWARE.

\documentclass{article}
\usepackage{../ssd}
\newcommand*\thetitle{Object Thinking}
\newcommand*\thesubtitle{and Domain Driven Design}
\begin{document}

\plush{\innoTitlePage{5}}

\plush{\pptQuote{faces/edsger-dijkstra}{Object-oriented programs are offered as alternatives to correct ones}{\nospell{Edsger W. Dijkstra} (1989)}}

\plush{\pptQuote{faces/alan-kay}{I invented the term \emph{object-oriented}, and I can tell you I did not have C++ in mind}{\nospell{Alan Kay} (1997)}}

\plush{\pptQuote{faces/paul-graham}{Object-oriented programming offers a sustainable way to write spaghetti code}{\nospell{Paul Graham} (2003)}}

\plush{\pptQuote{faces/linus-torvalds}{C++ is a horrible language. C++ leads to really, really bad design choices. ... idiotic object model crap.}{\nospell{Linus Torvalds} (2007)}}

\pptToc

\plush{\pptChapter[Theory]{The Philosophy of OOP}}

\plush[3]{%
  \pptSection[GOTO]{The Era of GOTO}
  \pptPic{0.7}{basic}
}

\plush[3]{%
  \pptSection[IF/THEN]{Structured Programming}
  \begin{multicols}{2}
    \pptPic{0.9}{pascal}
    \par\columnbreak
    \pptPic{0.7}{flowchart}
  \end{multicols}
}

\plush[3]{%
  \pptSection[CALL]{Procedural Programming}
  \pptPic{0.5}{calls}
}

\plush[3]{%
  \pptSection[OOP$_1$]{Object-Oriented Programming ... Not!}
  \pptPic{0.5}{oop-bad}
}

\plush[3]{%
  \pptBanner{Spaghetti OOP Code}
  \pptPic{0.8}{rectangles1}
}

\plush[3]{%
  \pptSection[OOP$_2$]{OOP Done Right}
  \pptPic{0.5}{oop-good}
}

\plush[3]{%
  \pptBanner{Elegant OOP Code}
  \pptPic{0.8}{rectangles2}
}

\plush{\pptChapter[Object]{What is an Object?}}

\plush{%
  \pptSection{C++}
  \pptQuote{books/stroustrup}{An object is some memory that holds a value of some type}{\emph{Programming Principles and Practice Using C++} by \nospell{Bjarne Stroustrup}}
}

\plush{%
  \pptSection{Wiki}
  \pptQuote{books/wikipedia}{Objects may contain data, in the form of fields, often known as attributes; and code, in the form of procedures, often known as methods}{Wikipedia}
}

\plush{%
  \pptSection{Smalltalk}
  \pptQuote{books/smalltalk}{An object consists of some private memory and a set of operations}{\emph{Smalltalk-80: The Language and Its Implementation} by \nospell{Adele Goldberg} et al., p.~6}
}

\plush{%
  \pptSection{Java}
  \pptQuote{books/java-nutshell}{A class is a collection of data fields that hold values and methods that operate on those values}{\emph{Java in a Nutshell} by \nospell{Ben Evans}}
}

\plush{%
  \pptSection{\nospell{Eckel}}
  \pptQuote{books/eckel}{Each object looks quite a bit like a little computer --- it has a state, and it has operations that you can ask it to perform}{\emph{Thinking in Java} by \nospell{Bruce Eckel}, p.~16}
}

\plush{%
  \pptSection{West}
  \pptQuote{books/object-thinking}{An object is the equivalent of the quanta from which the universe is constructed}{\emph{Object Thinking} by \nospell{David West}, p.~66}
}

\plush{\pptChapter[Evil]{Three Most Evil Parts of OOP}}

\plush[5]{%
  \pptPinQR{https://www.yegor256.com/2014/05/05/oop-alternative-to-utility-classes.html}
  \pptSection[Static]{1. Static Methods}
  \pptPic{0.5}{static1}
  \par
  \pptPic{0.5}{static2}
}

\plush[5]{%
  \pptPinQR{https://www.yegor256.com/2014/11/07/how-immutability-helps.html}
  \pptSection[Mutability]{2. Mutability vs. Immutability}
  \pptPic{0.5}{mutable}
  \par
  \pptPic{0.4}{immutable}
}

\plush[3]{%
  \pptPinQR{https://www.yegor256.com/2014/06/09/objects-should-be-immutable.html}
  \pptBanner{Benefits of Immutability}
  \pptPic{0.8}{immutability-benefits}
}

\plush[4]{%
  \pptPinQR{https://www.yegor256.com/2014/05/13/why-null-is-bad.html}
  \pptSection[NULL]{3. NULL References}
  \pptPic{0.8}{null}\par
  \emph{Null References, The Billion Dollar Mistake}\\
  presentation by \nospell{Tony Hoare},
  \href{http://www.infoq.com/presentations/Null-References-The-Billion-Dollar-Mistake-Tony-Hoare}{watch here}.
}

\plush[3]{%
  \pptBanner{NULL Object}
  \pptPic{0.8}{null-object}
}

\plush[7]{%
  \pptPinQR{https://www.yegor256.com/2015/08/25/fail-fast.html}
  \pptBanner{Fail Fast vs. Fail Safe}
  \pptPic{0.7}{fail-fast}
}

\plush{\pptChapter[DDD]{Domain Driven Design}}

\plush[4]{%
  \pptBanner{Names of Objects Done Right}
  \pptPic{0.6}{ddd}
}

\plush{\pptChapter{Elegant Objects}}

\print{\pptPinQR{https://www.elegantobjects.org}}
\plick{\pptBanner{Elegant Objects (EO)}}
\plick{Started in 2014}
\plick{Two books, 40+ speeches, 80+ blog posts}
\plick{30+ frameworks and libraries}
\plick{50+ fans registered}
\plick{Six bloggers\\\small e.g. \nospell{pragmaticobjects.com}, \nospell{g4s8.wtf}, \nospell{amihaiemil.com}}
\plush{Five ``Object Thinking'' Meetups}

\plush[3]{%
  \small
  \begin{multicols}{2}
  \emph{Object-Oriented Lies}\\
  JPoint Student Day\\
  Moscow, Russia, 22-24 April 2016\\
  \pptQR[1in]{https://www.youtube.com/watch?v=F4N25kZ2zQU}

  \emph{Java vs. OOP}\\
  JavaDay 2016 \\
  Minsk, Belarus, 11~June~2016\\
  \pptQR[1in]{https://www.youtube.com/watch?v=6hOBfjJ2bpw}

  \par\columnbreak

  \emph{Java vs. OOP}\\
  JavaDay Kyiv 2016 \\
  Kyiv, Ukraine, 15 October 2016\\
  \pptQR[1in]{https://www.youtube.com/watch?v=cGcCcxx4xrg}

  \emph{What's Wrong With OOP?}\\
  RigaDevDays 2017\\
  Riga, Latvia, 15 May 2017\\
  \pptQR[1in]{https://www.youtube.com/watch?v=K_QEOtYVQ7A}
  \end{multicols}
}

\plush[5]{%
  \pptPinQR{https://www.eolang.org}
  \pptBanner{EOLANG: Our New Programming Language}
  \pptPic{0.5}{eo-sample}
}

\print{\pptBanner{If you want to help:}}
\plick{EO to JavaScript/Go/Rust/Ruby compilers}
\plick{REPL for EO}
\plick{Static analysis of EO code}
\plick{EO integration with Java/C++}
\plick{Automated refactoring of EO code}
\plush{JetBrains plugin for EO}

\plush{\innoBVC}

\plush[3]{%
  \begin{multicols}{2}
    \innoBook{object-thinking}
      {David West}
      {Object Thinking}
    \par\columnbreak
    \innoBook{ddd}
      {Eric Evans}
      {Domain-Driven Design: Tackling Complexity in the Heart of Software}
  \end{multicols}
}

\plush[3]{%
  \begin{multicols}{2}
    \innoBook{elegant-objects-1}
      {Yegor Bugayenko}
      {Elegant Objects, vol. 1}
    \par\columnbreak
    \innoBook{elegant-objects-2}
      {Yegor Bugayenko}
      {Elegant Objects, vol. 2}
  \end{multicols}
}

\plush{%
  \pptBanner{Where to publish:}\par
  SPLASH: ACM SIGPLAN conference on Systems, Programming, Languages, and Applications\par
  PLDI: ACM SIGPLAN Conference on Programming Language Design and Implementation\par
  POPL: The Annual Symposium on Principles of Programming Languages
}

\plush[3]{%
  \pptBanner{Call to Action:}
  Take \ff{yegor256/hangman} repository and rewrite it in true
  object-oriented manner, submit a pull request with your changes.
}

\plush[3]{%
  \pptBanner[orange]{Still unresolved issues:}\par
  \begin{itemize}
    \item How to \ul{motivate} coders for better OO practices?
    \item How to \ul{create} better OO programming languages?
    \item How to \ul{catch} bad OO practices automatically?
    \item How to \ul{prove} some OO practices are bad?
  \end{itemize}
}

\end{document}
