% SPDX-FileCopyrightText: Copyright (c) 2021 Yegor Bugayenko
% SPDX-License-Identifier: MIT

\documentclass{article}
\usepackage{../ssd}
\newcommand*\thetitle{Serverless Design}
\newcommand*\thesubtitle{in Cloud}
\begin{document}

\lnTitlePage{12}{16}{FeMevj680tQ}

\plush[3]{\pptPic{0.8}{cloud}}

\pptToc

\plush{\pptChapter[AWS]{Amazon Web Services}}

\plush[5]{%
  \pptHeader[S3]{Simple Storage Service (S3)}
  \pptPic{0.7}{s3}
}

\plush[5]{%
  \pptHeader[EC2]{Elastic Compute Cloud (EC2)}
  \pptPic{0.7}{ec2}
}

\plush[5]{%
  \pptHeader{CloudFront}
  \pptPic{0.8}{cloudfront}
}

\plush[4]{%
  \pptHeader[SQS]{Simple Queue Service (SQS)}
  \pptPic{0.8}{sqs}
}

\plush[3]{%
  \pptHeader{Lambda}
  \pptPic{0.8}{lambda}
}

\plush[3]{%
  \pptHeader{OpenStack}
  \pptPic{0.8}{openstack}\par
  \url{https://docs.openstack.org/install-guide/get-started-logical-architecture.html}
}

\plush{\pptChapter[Docker]{Docker Containers}}

\plush[4]{%
  \pptHeader[Chef]{Chef, Puppet, etc.}
  \pptPic{0.8}{chef}
}

\plush[6]{%
  \pptHeader{Dockerfile}
  \begin{multicols}{2}
    \pptPic{0.8}{dockerfile}
    \columnbreak
    \pptPic{0.8}{dockerfile2}
  \end{multicols}
}

\plush[2]{%
  \pptHeader{LXC}
  \pptPic{0.7}{lxc}\par
  \small
  LXC is an operating-system-level virtualization method for running multiple isolated Linux systems on a control host using a single Linux kernel.
}

\plush[4]{%
  \pptHeader[K8s]{Kubernetes}
  \pptPic{0.7}{kubernetes}
}

\plush{\pptChapter[?aaS]{PaaS, IaaS, SaaS, EaaS, etc.}}

\plush[2]{%
  \pptHeader[EaaS]{Everything as a Service (EaaS)}
  \pptPic{0.5}{eaas}
}

\plush[5]{%
  \pptHeader[Heroku]{Platform as a Service (PaaS)}
  \pptPic{0.8}{heroku}
}

\plush[7]{%
  \pptHeader[Serverless]{Serverless Architecture}
  \pptPinQR{https://martinfowler.com/articles/serverless.html}
  Serverless architectures are application designs that incorporate third-party ``Backend as a Service'' (BaaS) services, and/or that include custom code run in managed, ephemeral containers on a ``Functions as a Service'' (FaaS) platform.\par
  Fundamentally, FaaS is about running backend code without managing \ul{your own server systems} or your own long-lived server applications.\par
  AWS RDS vs. AWS DynamoDB
}

\plush{\pptChapter[P2P]{P2P: BitTorrent, Blockchain, and Beyond}}

\plush[3]{%
  \pptHeader{BitTorrent}
  \pptPic{0.6}{bittorrent}
}

\plush[8]{%
  \pptHeader{Blockchain}
  \pptPic{0.5}{blockchain}
}

\plush{\innoBVC}

\plush[1]{%
  \begin{multicols}{2}
    \innoBook{aws}
      {Ben Piper}
      {AWS Certified Cloud Practitioner, Study Guide: CLF-C01 Exam}
    \par\columnbreak
    \innoBook{cloud}
      {\nospell{Thomas Erl} et al.}
      {Cloud Computing: Concepts, Technology \& Architecture}
  \end{multicols}
}

\plush[3]{%
  \pptBanner{Where to go:}
  AWS Certified Solution Architect
}

\plush[2]{%
  \pptBanner{Call to Action:}
  Use two AWS services in your app.
}

\plush[6]{%
  \pptBanner[orange]{Still unresolved issues:}
  \begin{itemize}
    \item How to \ul{simplify} serverless design?
    \item How to \ul{manage} data serverless?
    \item How to \ul{decentralize} better than Blockchain?
    \item How to \ul{enable} decentralized computing on IoT devices?
  \end{itemize}
}

\end{document}
