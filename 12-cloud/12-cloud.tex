% (The MIT License)
%
% Copyright (c) 2021 Yegor Bugayenko
%
% Permission is hereby granted, free of charge, to any person obtaining a copy
% of this software and associated documentation files (the 'Software'), to deal
% in the Software without restriction, including without limitation the rights
% to use, copy, modify, merge, publish, distribute, sublicense, and/or sell
% copies of the Software, and to permit persons to whom the Software is
% furnished to do so, subject to the following conditions:
%
% The above copyright notice and this permission notice shall be included in all
% copies or substantial portions of the Software.
%
% THE SOFTWARE IS PROVIDED 'AS IS', WITHOUT WARRANTY OF ANY KIND, EXPRESS OR
% IMPLIED, INCLUDING BUT NOT LIMITED TO THE WARRANTIES OF MERCHANTABILITY,
% FITNESS FOR A PARTICULAR PURPOSE AND NONINFRINGEMENT. IN NO EVENT SHALL THE
% AUTHORS OR COPYRIGHT HOLDERS BE LIABLE FOR ANY CLAIM, DAMAGES OR OTHER
% LIABILITY, WHETHER IN AN ACTION OF CONTRACT, TORT OR OTHERWISE, ARISING FROM,
% OUT OF OR IN CONNECTION WITH THE SOFTWARE OR THE USE OR OTHER DEALINGS IN THE
% SOFTWARE.

\documentclass{article}
\usepackage{../ssd}
\usepackage{../inno}
\usepackage{../slides}
\newcommand*\thetitle{Serverless Design}
\newcommand*\thesubtitle{in Cloud}
\begin{document}

\plush{\innoTitlePage{12}}

\plush[3]{\innoPic{0.8}{cloud}}

\innoToc

\plush{\innoChapter[AWS]{Amazon Web Services}}

\subcrumbection{S3}
\plush[5]{
  \innoHeader{Simple Storage Service (S3)}
  \innoPic{0.7}{s3}
}

\subcrumbection{EC2}
\plush[5]{
  \innoHeader{Elastic Compute Cloud (EC2)}
  \innoPic{0.7}{ec2}
}

\subcrumbection{CloudFront}
\plush[5]{
  \innoHeader{CloudFront}
  \innoPic{0.8}{cloudfront}
}

\subcrumbection{SQS}
\plush[4]{
  \innoHeader{Simple Queue Service (SQS)}
  \innoPic{0.8}{sqs}
}

\subcrumbection{Lambda}
\plush[3]{
  \innoHeader{Lambda}
  \innoPic{0.8}{lambda}
}

\subcrumbection{OpenStack}
\plush[3]{
  \innoHeader{OpenStack}
  \innoPic{0.8}{openstack}\par
  \url{https://docs.openstack.org/install-guide/get-started-logical-architecture.html}
}

\plush{\innoChapter[Docker]{Docker Containers}}

\subcrumbection{Chef}
\plush[4]{
  \innoHeader{Chef, Puppet, etc.}
  \innoPic{0.8}{chef}
}

\subcrumbection{Dockerfile}
\plush[6]{
  \innoHeader{Dockerfile}
  \begin{multicols}{2}
    \innoPic{0.8}{dockerfile}
    \columnbreak
    \innoPic{0.8}{dockerfile2}
  \end{multicols}
}

\subcrumbection{LXC}
\plush[2]{
  \innoHeader{LXC}
  \innoPic{0.7}{lxc}\par
  LXC is an operating-system-level virtualization method for running multiple isolated Linux systems on a control host using a single Linux kernel.
}

\subcrumbection{K8s}
\plush[4]{
  \innoHeader{Kubernetes}
  \innoPic{0.7}{kubernetes}
}

\plush{\innoChapter[?aaS]{PaaS, IaaS, SaaS, EaaS, etc.}}

\subcrumbection{EaaS}
\plush[2]{
  \innoHeader{Everything as a Service (EaaS)}
  \innoPic{0.5}{eaas}
}

\subcrumbection{Heroku}
\plush[5]{
  \innoHeader{Platform as a Service (PaaS)}
  \innoPic{0.8}{heroku}
}

\subcrumbection{Serverless}
\plush[7]{
  \innoHeader{Serverless Architecture}
  \innoPinQR{https://martinfowler.com/articles/serverless.html}
  Serverless architectures are application designs that incorporate third-party ``Backend as a Service'' (BaaS) services, and/or that include custom code run in managed, ephemeral containers on a ``Functions as a Service'' (FaaS) platform.\par
  Fundamentally, FaaS is about running backend code without managing \ul{your own server systems} or your own long-lived server applications.\par
  AWS RDS vs. AWS DynamoDB
}

\plush{\innoChapter[P2P]{P2P: BitTorrent, Blockchain, and Beyond}}

\subcrumbection{BitTorrent}
\plush[3]{
  \innoHeader{BitTorrent}
  \innoPic{0.6}{bittorrent}
}

\subcrumbection{Blockchain}
\plush[8]{
  \innoHeader{Blockchain}
  \innoPic{0.5}{blockchain}
}

\plush{\innoBVC}

\plush[1]{%
  \begin{multicols}{2}
    \innoBook{aws}
      {Ben Piper}
      {AWS Certified Cloud Practitioner, Study Guide: CLF-C01 Exam}
    \par\columnbreak
    \innoBook{cloud}
      {\nospell{Thomas Erl} et al.}
      {Cloud Computing: Concepts, Technology \& Architecture}
  \end{multicols}
}

\plush[3]{%
  \innoBanner{Where to go:}
  AWS Certified Solution Architect
}

\plush[2]{%
  \innoBanner{Call to Action:}
  Use two AWS services in your app.
}

\plush[6]{%
  \innoBanner[orange]{Still unresolved issues:}
  \begin{itemize}
    \item How to \ul{simplify} serverless design?
    \item How to \ul{manage} data serverless?
    \item How to \ul{decentralize} better than Blockchain?
    \item How to \ul{enable} decentralized computing on IoT devices?
  \end{itemize}
}

\end{document}
