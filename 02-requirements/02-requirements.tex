% (The MIT License)
%
% Copyright (c) 2021 Yegor Bugayenko
%
% Permission is hereby granted, free of charge, to any person obtaining a copy
% of this software and associated documentation files (the 'Software'), to deal
% in the Software without restriction, including without limitation the rights
% to use, copy, modify, merge, publish, distribute, sublicense, and/or sell
% copies of the Software, and to permit persons to whom the Software is
% furnished to do so, subject to the following conditions:
%
% The above copyright notice and this permission notice shall be included in all
% copies or substantial portions of the Software.
%
% THE SOFTWARE IS PROVIDED 'AS IS', WITHOUT WARRANTY OF ANY KIND, EXPRESS OR
% IMPLIED, INCLUDING BUT NOT LIMITED TO THE WARRANTIES OF MERCHANTABILITY,
% FITNESS FOR A PARTICULAR PURPOSE AND NONINFRINGEMENT. IN NO EVENT SHALL THE
% AUTHORS OR COPYRIGHT HOLDERS BE LIABLE FOR ANY CLAIM, DAMAGES OR OTHER
% LIABILITY, WHETHER IN AN ACTION OF CONTRACT, TORT OR OTHERWISE, ARISING FROM,
% OUT OF OR IN CONNECTION WITH THE SOFTWARE OR THE USE OR OTHER DEALINGS IN THE
% SOFTWARE.

\documentclass[static]{../slidedeck}
\newcommand*\thetitle{Requirements}
\newcommand*\thesubtitle{Engineering}
\begin{document}

\innoTitlePage{2}

\sdPrintAndFlush{\sdQuote{../03-rup/agile-martin}{If you are lucky, you start a project with a clear picture of what you want the system to be. The design of the system is a vital image in your mind. If you are luckier still, the clarity of that design makes it to the first release.}{Agile Software Development. Principles, Patterns, and Practices \br Robert Martin}}

\sdTocPrint

\sdToc{UCs}{Use Cases (user stories)}

\sdPrintAndFlush{
  \begin{multicols}{2}
    \sdPic{0.4}{jacobson}\br
    Ivar Jacobson \br
    in 1987

    \columnbreak

    \sdPic{0.3}{booch}\br
    Grady Booch

    \sdPic{0.3}{rumbaugh}\br
    James Rumbaugh
  \end{multicols}
  }

\sdPrintAndFlush{\sdQuote{uml-book}{There is no standard way to write the content of a use case, and different formats work well in different cases}{UML Distilled \br Martin Fowler}}

\begin{ffcode}
Use Case: Make a QR code
Primary Actor: User
Basic flow:
  1. The User enters the URL into the HTML box.
  2. The System creates a PNG image of a QR code.
  3. The User downloads the image through HTTP.
Extensions:
  1a. The User enters broken URL.
    1. The System shows an error modal dialog box.
    2. The User confirms.
  3a. The User cancels downloading.
    1. The System stops sending data over HTTP.
\end{ffcode}

\sdPrintAndFlush{
  \sdBanner{Use Case Diagram}

  \colorbox{sd-white}{\sdPic{0.5}{uc-diagram}}
}

\sdToc{FPA}{FPA \& IFPUG \& COSMIC}

\sdPrint{\sdBanner{Function Point Analysis (FPA)}}
\sdPrint{FPA was originally developed by Allan Albrecht in the late 1970s at IBM}\sdClick
\sdPrint{International Function Point Users Group (IFPUG) is in charge}\sdClick
\sdPrint{Regulated by ISO 20296}\sdClick
\sdPrint{COSMIC is the modern version: ISO/IEC 14143}\sdClick
\sdPrint{The method breaks down the requirements into combinations of the four data movements types: Entry (E), Exit (X), Read (R), Write (W)}\sdClick
\sdPrint{Function Points (FPs) are used for estimates}
\sdFlush

\sdPrintAndFlush{
  \sdQR{https://en.wikipedia.org/wiki/Use_case_points}

  Also, read about Use Case Points (UCP)
}

\sdToc{Matrix}{Traceability Matrix}

\sdPrintAndFlush{
  \sdBanner{Traceability Matrix}

  \colorbox{sd-white}{\sdPic{0.6}{matrix}}

  {\small Use Cases, Non-Functional Requirements, Test Cases,
  Classes, Packages, Servers, Containers, Components,
  Nodes, etc.}
}

\sdToc{V\&V}{Verification \& Validation}

\sdPrintAndFlush{\sdQuote{cmmi-vv}{Two different questions may be asked by a project team:
\ul{`We do it right?'} vs. \ul{`We do the right thing?'} CMMI-Dev has two separate process areas: VER and VAL.}{}}

\sdToc{NFRs}{Non-Functional Requirements (NFRs)}

\sdPrint{\sdBanner{Quality Attributes or just ``-ilities'':}}
\sdPrint{Availability: $A = \dfrac{E_\text{up}}{E_\text{down} + E_\text{up}}$}\sdClick
\sdPrint{Capacity: Clicks Per Second (CPS)}\sdClick
\sdPrint{Recovery: Recovery Time Objective (RTO)}\sdClick
\sdPrint{Maintainability: Mean Time To Repair (MTTR)}\sdClick
\sdPrint{Usability: focus group surveys?}
\sdFlush

\sdPrintAndFlush{
  \begin{multicols}{2}
    \sdBanner[red]{Bad NFRs}

    ``The software must be fast.''

    ``It must be easy to maintain.''

    ``The user interface must be attractive.''

    \columnbreak
    \small

    \sdBanner{Good NFRs}
    ``Being installed on a test server (see Annex A),
    the software must respond is less than 20ms
    on any request from UC1--UC7.''

    ``The maximum time required to fix a bug
    much be less than two hours.''

    ``At least 80\% of beta users must anonymously
    confirm that the UI is attractive enough.''
  \end{multicols}
}

\sdToc{COCOMO}{Estimates and COCOMO II}

\sdPrintAndFlush{\sdQuote{boehm}{For software decisions, the most critical and difficult of these inputs to provide are estimates of the cost of a proposed software project.}{Software Engineering Economics \br Barry W. Boehm}}

\sdPrintAndFlush{
  First, we predict the size in Kilo Lines of Code ($K$).

  Then, we find effort adjustment factor ($F$).

  Then, we find coefficients $a$, $b$, and $c$ using the table.

  Then, we calculate the size in man-months:

  $$E = a \times K^b \times F$$

  Then, we calculate the duration in months ($D$):

  $$D = 2.5 \times E^c$$
}

\sdPrintAndFlush{
  \sdBanner{Cone of Uncertainty}
  \sdPic{0.6}{cone}
}

\sdPrintAndFlush{
  \sdQR{https://www.yegor256.com/2015/06/02/how-to-estimate-software-cost.html}

  ``How much for this sofware?'' they ask.

  ``The development will take forever and will consume all your money,'' we answer.
}

\sdToc{B.V.C}{Books, Venues, Call-to-Action}

\sdPrintAndFlush{
  \begin{multicols}{2}
    \innoBook{wiegers}
      {Karl Wiegers et al.}
      {Software Requirements}
    \columnbreak
    \innoBook{wiegers2}
      {Karl Wiegers}
      {More About Software Requirements: Thorny Issues and Practical Advice}
  \end{multicols}
}

\sdPrintAndFlush{
  \begin{multicols}{2}
    \innoBook{cockburn}
      {Alistair Cockburn}
      {Writing Effective Use Cases}
    \columnbreak
    \innoBook{mcconnell}
      {Steve McConnell}
      {Software Estimation: Demystifying the Black Art}
  \end{multicols}
}

\sdPrintAndFlush{
  \sdBanner{Where to publish:}

  IEEE International Requirements Engineering Conference (RE)
}

\sdPrintAndFlush{
  \sdBanner{Call to Action:}

  Specify 4+ use cases and 10+ non-functional requirements for your app.
  Then, count all FPs, estimate the size in LoC, and then estimate
  the cost in man-months using COCOMO II.
}

\sdPrintAndFlush{
  \sdBanner[orange]{Still unresolved issues:}

  \begin{itemize}
    \item How to \ul{validate} requirements automatically?
    \item How to \ul{trace} them automatically?
    \item How to \ul{specify} in Controlled Natural Language (CNL)?
    \item How to \ul{reduce} them?
  \end{itemize}
}

\end{document}
