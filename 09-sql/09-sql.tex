% SPDX-FileCopyrightText: Copyright (c) 2021 Yegor Bugayenko
% SPDX-License-Identifier: MIT

\documentclass{article}
\usepackage{../ssd}
\newcommand*\thetitle{IDEF}
\newcommand*\thesubtitle{and SQL/NoSQL Databases}
\begin{document}

\plush{\innoTitlePage{9}}

\pptToc

\plush{\pptChapter[Types]{Types of Databases}}

\plush[2]{%
  \pptSection[Navi]{Navigational Databases}
  \pptPic{0.8}{navigational}\par
  E.g. Integrated Data Store (IDS)
}

\plush[2]{%
  \pptSection[Hierarchical]{Hierarchical Databases}
  \pptPic{0.8}{hierarchical}\par
  E.g. IBM Information Management System (IMS)
}

\plush[5]{%
  \pptSection[SQL]{Relational Databases (SQL)}
  \pptPic{0.8}{relational}\par
  E.g. Oracle Database
}

\plush[3]{%
  \pptSection[Object]{Object Databases}
  \pptPic{0.8}{object}\par
  E.g. ObjectStore
}

\plush[4]{%
  \pptSection[K-V]{Key-Value Databases}
  \pptPic{0.4}{key-value}\par
  \pptPic{0.5}{dynamo}\par
  E.g. Redis, AWS DynamoDB
}

\plush[3]{%
  \pptSection[Columnar]{Columnar Databases}
  \pptPic{0.8}{columnar}\par
  E.g. ClickHouse by Yandex\par
  Read also about Online Analytical Processing (OLAP)
}

\plush[4]{%
  \pptSection[NoSQL]{Document Databases (a.k.a. NoSQL)}
  \pptPic{0.5}{document}\par
  E.g. MongoDB
}

\plush[4]{%
  \pptSection[XML]{XML Databases}
  \pptPic{0.4}{xml}\par
  E.g. eXist
}

\plush[5]{%
  \pptSection[S2]{AWS S3}
  \pptPic{0.9}{s3}\par
}

\plush[4]{%
  \pptSection[Graph]{Graph Databases}
  \pptPic{0.5}{graph}\par
  E.g. Neo4j
}

\plush[2]{%
  \pptSection{NewSQL}
  ``NewSQL is a class of relational database management systems that seek to provide the scalability of NoSQL systems for online transaction processing (OLTP) workloads while maintaining the ACID guarantees of a traditional database system'' --- Wikipedia\par
  E.g. VoltDB
}

\plush[5]{%
  \pptSection[Big]{Big Data}
  \pptPic{0.8}{hadoop}\par
  E.g. Hadoop
}

\plush{\pptChapter[ER]{IDEF1X, ER Model, UML}}

\plush[3]{%
  \pptSection[IDEF1X]{IDEF1X Model for Databases}
  \pptPic{0.6}{idef1x-diagram}
}

\plush[3]{%
  \pptSection[UML]{UML Class Diagram as Database Model}
  \pptPic{0.8}{uml}
}

\plush[3]{%
  \pptSection[ER]{Entity-Relationship (ER) Model}
  \pptPic{0.8}{er-diagram}
}

\plush{\pptChapter[Properties]{How to Choose the Right Database?}}

\print{\pptSection{ACID}}
\plick{\nospell{\textbf{\large A}tomicity}: everything or nothing}
\plick{\nospell{\textbf{\large C}onsistency}: invariants are in place}
\plick{\nospell{\textbf{\large I}solation}: concurrent or sequential}
\plush[5]{\nospell{\textbf{\large D}urability}: completed transactions $\rightarrow$ non-volatile memory}

\print{\pptSection[Speed]{Performance}}
\plick{Queries Profiling \& Optimization}
\plick{Denormalization}
\plush[3]{Caching}

\print{\pptSection[Scale]{Scalability}}
\plick{Vertical vs. Horizontal Scalability}
\plick{Sharding vs. Master-Slave Replication}
\plush[5]{\pptPic{0.8}{sharding}}

\print{\pptSection[Libraries]{Application Layer Support}}
\plick{Is it open source?}
\plick{How mature is the library?}
\plick{Is it a thin driver or ORM-ish framework?}
\plick{How many languages are supported?}
\plick{Are there alternatives?}
\plush[6]{Is the API open?}

\plush[5]{%
  \pptSection[Versions]{Schema Version Control}
  \pptPic{0.8}{0rsk}\par
  \small\url{https://github.com/yegor256/0rsk/blob/master/liquibase}\par
  Liquibase, Flyway, Mongobee, Mongock, etc.
}

\plush[4]{%
  \pptSection[Integrity]{Integrity: Primary and Foreign Keys}
  \pptPic{0.8}{keys}
}

\plush[3]{%
  \pptSection{Resilience}
  \pptPic{0.7}{mongo-failure}\par
  Resilience is the capacity of your database infrastructure to \ul{recover} from disaster and keep on providing service.
}

\plush[2]{%
  \pptSection[Liability]{Liability: Who Pays for Losses?}
  AWS Terms, \href{https://aws.amazon.com/agreement/}{Section 11}: "WE AND OUR AFFILIATES AND \nospell{LICENSORS} WILL NOT BE LIABLE TO YOU FOR ANY INDIRECT, INCIDENTAL, SPECIAL, CONSEQUENTIAL OR EXEMPLARY DAMAGES. INCLUDING DAMAGES FOR LOSS OF PROFITS, REVENUES, CUSTOMERS, OPPORTUNITIES, GOODWILL, USE, \ul{OR DATA}."
}

\plush[3]{%
  \pptSection[Durability]{Durability: Can We Loose Data?}
  \href{https://aws.amazon.com/s3/faqs}{FAQ}: Amazon S3 is designed to provide 99.999999999\% (11 9's) of data durability of objects over a given year. This durability level corresponds to an average annual expected loss of 0.000000001\% of objects. For example, if you store 10,000,000 objects with Amazon S3, you can on average expect to incur a loss of a single object once every 10,000 years.''
}

\plush{\innoBVC}

\plush[1]{%
  \begin{multicols}{2}
    \innoBook{date}
      {C.J. Date}
      {An Introduction to Database Systems, 8th Edition}
    \par\columnbreak
    \innoBook{nosql-distilled}
      {\nospell{Pramod Sadalage} et al.}
      {NoSQL Distilled: A Brief Guide to the Emerging World of Polyglot Persistence}
  \end{multicols}
}

\plush[1]{%
  \pptBanner{Where to go:}
  ACM SIGMOD/PODS Conference.
}

\plush[2]{%
  \pptBanner{Call to Action:}
  Design a relational persistence layer in your app and then
  make it NoSQL; compare pros and cons.
}

\plush[4]{%
  \pptBanner[orange]{Still unresolved issues:}
  \begin{itemize}
    \item How to \ul{guarantee} 100\% durability, etc.?
    \item How to \ul{optimize} database schema automatically?
    \item How to \ul{generate} optimal schema automatically?
    \item How to \ul{make} object databases popular?
  \end{itemize}
}

\end{document}
