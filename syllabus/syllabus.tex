% (The MIT License)
%
% Copyright (c) 2021 Yegor Bugayenko
%
% Permission is hereby granted, free of charge, to any person obtaining a copy
% of this software and associated documentation files (the 'Software'), to deal
% in the Software without restriction, including without limitation the rights
% to use, copy, modify, merge, publish, distribute, sublicense, and/or sell
% copies of the Software, and to permit persons to whom the Software is
% furnished to do so, subject to the following conditions:
%
% The above copyright notice and this permission notice shall be included in all
% copies or substantial portions of the Software.
%
% THE SOFTWARE IS PROVIDED 'AS IS', WITHOUT WARRANTY OF ANY KIND, EXPRESS OR
% IMPLIED, INCLUDING BUT NOT LIMITED TO THE WARRANTIES OF MERCHANTABILITY,
% FITNESS FOR A PARTICULAR PURPOSE AND NONINFRINGEMENT. IN NO EVENT SHALL THE
% AUTHORS OR COPYRIGHT HOLDERS BE LIABLE FOR ANY CLAIM, DAMAGES OR OTHER
% LIABILITY, WHETHER IN AN ACTION OF CONTRACT, TORT OR OTHERWISE, ARISING FROM,
% OUT OF OR IN CONNECTION WITH THE SOFTWARE OR THE USE OR OTHER DEALINGS IN THE
% SOFTWARE.

\documentclass[nobrand,anonymous,nodate,nosecurity]{huawei}
\begin{document}

{\sffamily{\bfseries\Large Software System Design}\\
16-lectures course, the syllabus}

Prerequisites to the course (it is expected that a student knows this):

\begin{itemize}
\item How to code
\item How to use Git
\end{itemize}

After the course a student \emph{hopefully} will know:

\begin{itemize}
\item How to manage software requirements
\item How to develop iteratively and incrementally
\item How to think with objects, not procedures
\item How to use design patterns and not use anti-patterns
\item How to draw and share knowledge using UML
\item How to choose and use data formats, e.g. XML or JSON
\item How to choose a database management server
\item How to deploy software continuously
\item How to build distributed software systems
\item How to test software
\item How to measure the quality of software design
\end{itemize}

At the end of the course a student receives a \emph{score} of up
to 100 points. The points are given after a \emph{subjective} review
of an open source software product created by the student during the
course (no oral presentation is needed).
Even though the review is subjective, the following
balance has to be maintained (the questions provided below
stand merely as examples and do not constitute the entire scope):

\begin{itemize}
\item {\bfseries\scshape Requirements} (15\%):
  Glossary is in place?
  Stakeholders and their concerns are identified?
  Use cases explain functional requirements?
  Non-functional requirements are documented?
  NFRs are measurable?
\item {\bfseries\scshape Design} (25\%):
  UML diagrams, such as Class, Component, Deployment, and Sequence, are present?
  Design decisions are explained?
  Design patterns are used?
  Traceability between requirements and design elements is visible?
\item {\bfseries\scshape Architecture} (30\%):
  The design is modular?
  The composition of modules makes sense?
  Design elements are cohesive?
  Design elements are decoupled enough?
  The build is automated?
  The delivery pipeline is automated?
\item {\bfseries\scshape Code} (15\%):
  The code is clean enough?
  In-code documentation is present?
  Static analyzers and style checkers are used?
  Unit tests are in place?
  Integration tests are present?
  Is test coverage being measured?
\item {\bfseries\scshape Spirit} (15\%):
  The product is somewhat popular on GitHub (or a similar platform)?
  Issues and pull requests were used during development?
  Commit comments are detailed enough?
  GitHub features are actively used, like releases, actions, etc.?
\end{itemize}

A few versions of the product may be presented for review:
Alpha, Beta, and Final. The scores given to a student
after version reviews don't affect the overall
score given at the end of the course. However, if Alpha
version is not delivered, a student gets a penalty of 10 negative points,
while a missed Beta gives 20 negative points. Thus, if a student
ignores both versions and brings a great product at the end of the course,
he or she gets $100-30=70$ points at most.

The score may be turned into a grade using the following formula:

\begin{itemize}
\item \textbf{A} Excellent: 90+
\item \textbf{B} Good: 75+
\item \textbf{C} Satisfactory: 55+
\item \textbf{D} Poor: 0+
\end{itemize}

The following books are highly recommended to read (in no particular order):

\begin{multicols}{2}\small\raggedright
Len Bass et al., \emph{Software Architecture in Practice}\\
Paul Clements et al., \emph{Documenting Software Architectures: Views and Beyond}\\
\nospell{Karl Wiegers} et al., \emph{Software Requirements}\\
{\nospell{Alistair Cockburn}}, \emph{Writing Effective Use Cases}\\
{\nospell{Steve McConnell}}, \emph{Software Estimation: Demystifying the Black Art}\\
{Robert Martin}, \emph{Clean Architecture: A Craftsman's Guide to Software Structure and Design}\\
{Steve McConnell}, \emph{Code Complete}\\
{Frederick Brooks Jr.}, \emph{Mythical Man-Month, The: Essays on Software Engineering}\\
{David Thomas et al.}, \emph{The Pragmatic Programmer: Your Journey To Mastery}\\
{Robert C. Martin}, \emph{Clean Code: A Handbook of Agile Software Craftsmanship}\\
{\nospell{Grady Booch} et al.}, \emph{Object-Oriented Analysis and Design with Applications}\\
{\nospell{Bjarne Stroustrup}}, \emph{Programming: Principles and Practice Using C++}\\
{\nospell{Brett McLaughlin} et al.}, \emph{Head First Object-Oriented Analysis and Design: A Brain Friendly Guide to OOA\&D}\\
{David West}, \emph{Object Thinking}\\
{Eric Evans}, \emph{Domain-Driven Design: Tackling Complexity in the Heart of Software}\\
{Yegor Bugayenko}, \emph{Elegant Objects}\\
{Michael Feathers}, \emph{Working Effectively with Legacy Code}\\
{Martin Fowler}, \emph{Refactoring: Improving the Design of Existing Code}\\
{Erich Gamma et al.}, \emph{Design Patterns: Elements of Reusable Object-Oriented Software}\\
{Scott Meyers}, \emph{Effective C++: 55 Specific Ways to Improve Your Programs and Designs}\\
{\nospell{Elliotte Rusty Harold} et al.}, \emph{XML in a Nutshell, Third Edition}\\
{\nospell{Michael James Fitzgerald}}, \emph{Learning XSLT: A Hands-On Introduction to XSLT and XPath}\\
{Martin Fowler}, \emph{UML Distilled}\\
{\nospell{Anneke Kleppe} et al.}, \emph{MDA Explained: The Model Driven Architecture: Practice and Promise}\\
{C.J. Date}, \emph{An Introduction to Database Systems, 8th Edition}\\
{\nospell{Pramod Sadalage} et al.}, \emph{NoSQL Distilled: A Brief Guide to the Emerging World of Polyglot Persistence}\\
{\nospell{Jez Humble} et al.}, \emph{Continuous Delivery: Reliable Software Releases through Build, Test, and Deployment Automation}\\
{\nospell{Michael T. Nygard}}, \emph{Release It!: Design and Deploy Production-Ready Software}\\
{Leonard Richardson et al.}, \emph{RESTful Web APIs: Services for a Changing World}
\end{multicols}

\end{document}
