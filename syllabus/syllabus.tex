% (The MIT License)
%
% Copyright (c) 2021 Yegor Bugayenko
%
% Permission is hereby granted, free of charge, to any person obtaining a copy
% of this software and associated documentation files (the 'Software'), to deal
% in the Software without restriction, including without limitation the rights
% to use, copy, modify, merge, publish, distribute, sublicense, and/or sell
% copies of the Software, and to permit persons to whom the Software is
% furnished to do so, subject to the following conditions:
%
% The above copyright notice and this permission notice shall be included in all
% copies or substantial portions of the Software.
%
% THE SOFTWARE IS PROVIDED 'AS IS', WITHOUT WARRANTY OF ANY KIND, EXPRESS OR
% IMPLIED, INCLUDING BUT NOT LIMITED TO THE WARRANTIES OF MERCHANTABILITY,
% FITNESS FOR A PARTICULAR PURPOSE AND NONINFRINGEMENT. IN NO EVENT SHALL THE
% AUTHORS OR COPYRIGHT HOLDERS BE LIABLE FOR ANY CLAIM, DAMAGES OR OTHER
% LIABILITY, WHETHER IN AN ACTION OF CONTRACT, TORT OR OTHERWISE, ARISING FROM,
% OUT OF OR IN CONNECTION WITH THE SOFTWARE OR THE USE OR OTHER DEALINGS IN THE
% SOFTWARE.

\documentclass[nobrand,anonymous,nodate,nosecurity]{huawei}
\begin{document}

{\sffamily{\large Software System Design}\\
16-lectures course, the Syllabus}

Prerequisites to the course (it is expected that a student knows this):

\begin{itemize}
\item How to code
\item How to use Git
\end{itemize}

After the course a student \emph{hopefully} will know:

\begin{itemize}
\item How to manage software requirements
\item How to develop iteratively and incrementally
\item How to think with objects, not procedures
\item How to use design patterns and not use anti-patterns
\item How to draw and share knowledge using UML
\item How to choose and use data formats, e.g. XML or JSON
\item How to choose a database management server
\item How to deploy software continuously
\item How to build distributed software systems
\item How to test software
\item How to measure the quality of software design
\end{itemize}

At the end of the course a student receives a \emph{score} of up
to 100 points. The points are given after a \emph{subjective} review
of an open source software product created by the student during the
course. Even though the review is subjective, the following
balance has to be maintained (the questions provided below
stand merely as examples and do not constitute the entire scope):

\begin{itemize}
\item {\scshape Requirements} (15\%):
  Glossary is in place?
  Use cases are used to explain functional requirements?
  Non-functional requirements are explained?
  NFRs are measurable?
\item {\scshape Design} (25\%):
  UML diagrams, such as component, deployment, class, and sequence, are present?
  Design decisions are explained?
  Design patterns are used?
  Traceability between requirements and design elements is visible?
\item {\scshape Architecture} (30\%):
  The design is modular?
  The composition of modules make sense?
  Design elements are cohesive?
  Build is automated?
  Delivery pipeline is automated?
\item {\scshape Code} (15\%):
  The code is clean?
  In-code documentation is present?
  Static analyzers and style checkers are used?
  Unit tests are in place?
  Integration tests?
\item {\scshape Spirit} (15\%):
  Is product already somewhat popular on GitHub (or similar platform)?
  Issues and pull requests are used?
  Commit comments are detailed enough?
  GitHub features are actively used, like releases, actions, etc.?
\end{itemize}

A few versions of the product may be presented for review:
Alpha, Beta, and Final. The scores given to a student
after version reviews don't affect the overall
score given at the end of the course.

\end{document}
