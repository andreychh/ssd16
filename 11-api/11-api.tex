% (The MIT License)
%
% Copyright (c) 2021 Yegor Bugayenko
%
% Permission is hereby granted, free of charge, to any person obtaining a copy
% of this software and associated documentation files (the 'Software'), to deal
% in the Software without restriction, including without limitation the rights
% to use, copy, modify, merge, publish, distribute, sublicense, and/or sell
% copies of the Software, and to permit persons to whom the Software is
% furnished to do so, subject to the following conditions:
%
% The above copyright notice and this permission notice shall be included in all
% copies or substantial portions of the Software.
%
% THE SOFTWARE IS PROVIDED 'AS IS', WITHOUT WARRANTY OF ANY KIND, EXPRESS OR
% IMPLIED, INCLUDING BUT NOT LIMITED TO THE WARRANTIES OF MERCHANTABILITY,
% FITNESS FOR A PARTICULAR PURPOSE AND NONINFRINGEMENT. IN NO EVENT SHALL THE
% AUTHORS OR COPYRIGHT HOLDERS BE LIABLE FOR ANY CLAIM, DAMAGES OR OTHER
% LIABILITY, WHETHER IN AN ACTION OF CONTRACT, TORT OR OTHERWISE, ARISING FROM,
% OUT OF OR IN CONNECTION WITH THE SOFTWARE OR THE USE OR OTHER DEALINGS IN THE
% SOFTWARE.

\documentclass{article}
\usepackage{../ssd}
\usepackage{../inno}
\usepackage{../slides}
\newcommand*\thetitle{Microservices}
\newcommand*\thesubtitle{and RESTful APIs}
\begin{document}

% gRPC
% SOAP + autogenerated stubs
% EJB
% RMI
% IoT

\plush{\innoTitlePage{11}}

\innoToc

\plush{\innoChapter[SOA]{Service-Oriented Architecture (SOA)}}

\plush{\innoPic{0.9}{soa}}

\subcrumbection{RPC}
\plush{
  \innoSection{XML RPC}
  \innoPic{0.5}{rpc}
}

\subcrumbection{SOAP}
\plush{
  \innoSection{Simple Object Access Protocol (SOAP)}
  \innoPic{0.9}{soap}
}

\subcrumbection{CORBA}
\plush{
  \innoSection{Common Object Request Broker Architecture (CORBA)}
  \innoPic{0.9}{corba}
}

\subcrumbection{IDL}
\plush{
  \innoSection{Interface Description Language (IDL)}
  \innoPic{0.9}{idl}
}

\plush{\innoChapter[RESTful]{Microservices and RESTful API}}

\subcrumbection{Microservices}
\plush{
  \innoSection{Microservices}
  \innoPic{0.9}{microservices}\par
  ``Microservices are a modern interpretation of service-oriented
  architectures used to build distributed software systems." --- Wikipedia
}

\plush{
  \innoBanner{Stateless vs. Stateful Architecture}
  ``A stateless process or application can be understood in isolation. There is no stored knowledge of or reference to past transactions. Each transaction is made as if from scratch for the first time.'' ---
  \href{https://www.redhat.com/en/topics/cloud-native-apps/stateful-vs-stateless}{RedHat}
}

\subcrumbection{REST}
\plush{
  \innoSection{Representational State Transfer (REST)}
  \innoPic{0.8}{restful}
  \innoPic{0.8}{restful2}
}

\subcrumbection{HATEOAS}
\plush{
  \innoSection{Hypermedia As The Engine Of Application State (HATEOAS)}
  \innoPic{0.8}{hateoas}
}

\subcrumbection{Mesh}
\plush{
  \innoSection{Service Mesh}
  \innoPic{0.8}{mesh}
  ``A service mesh is a dedicated infrastructure layer for facilitating service-to-service communications between services or microservices, using a proxy'' --- Wikipedia
}

\plush{\innoChapter[RESTful]{Protobuf and gRPC}}

\subcrumbection{gRPC}
\plush{
  \innoSection{gRPC}
  \innoPic{0.8}{grpc}
}

\subcrumbection{Protobuf}
\plush{
  \innoSection{Protobuf}
  \begin{multicols}{2}
    \innoPic{0.5}{proto1}
    \par\columnbreak
    \innoPic{0.5}{proto2}
  \end{multicols}
}

\plush{\innoPic{0.8}{grpc-joke}}

\plush{\innoChapter[IoT]{Internet of Things (IoT)}}

\innoBVC

\plush[2]{
  \begin{multicols}{2}
    \innoBook{restful}
      {Leonard Richardson et al.}
      {RESTful Web APIs: Services for a Changing World}
    \par\columnbreak
    \innoBook{}
      {..}
      {..}
  \end{multicols}
}

\plush[5]{
  \innoBanner{Where to go:}
  ..
}

\plush[4]{
  \innoBanner{Call to Action:}
  ...
}

\plush[4]{
  \innoBanner[orange]{Still unresolved issues:}
  \begin{itemize}
    \item How to \ul{reverse} code to a model?
    \item How to \ul{sync} a model with the code?
    \item How to \ul{simplify} UML for practical programming?
    \item How to \ul{restore} faith in MDA?
  \end{itemize}
}

\end{document}
